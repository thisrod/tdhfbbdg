\input respnotes
\input xpmath
\input unifonts \tenrm

\def\cite#1{{\tt #1}}

\title Dimensionless units

Start with some identities for Gaussians.  The normal distribution
with standard deviation~$σ(t)$ is
$$φ(x,t) = {e^{-x²/2σ²}\over √{2π}σ}.$$
This is normalised with~$∫φ=1$.  It satisfies
$$φ_t = \left({x²\over σ²}-1\right){φ\over σ}\dot σ
	\qquad{\rm and}\qquad
	φ_{xx} = \left({x²\over σ²}-1\right){φ\over σ²}.
$$
So the heat equation~$φ_t=φ_{xx}$ has solution~$σ(t)=√{2t}$, which
satisfies~$\dot σ={1\over σ}$.

The corresponding wave function is
$$ψ(x) = {e^{-x²/2σ²}\over √{π}σ},$$
with~$∫ψ²=1$.  This satisfies
$$ψ_x = -{x\over σ²}ψ
	\qquad{\rm and}\qquad
	ψ_{xx} = {1\over σ²}\left({x²\over σ²}-1\right)ψ.
$$

I have been using a 1D harmonic oscillator Hamiltonian of~$H=-D_{xx}+x²$.
The normal choice is~$H=½(-D_{xx}+x²)$, obtained by setting~$ℏ=m$
and~$ω=1$.  The eigenstate is clearly the same in both cases,
with~$σ=1$.  This gives~$ψ₀(x)=π^{-1\over 4}e^{-x²/2}$, and, in 2D,
$ψ₀(x,y)=ψ₀(x)·ψ₀(y)={1\over √π}e^{-(x²+y²)/2}$.  The normal
hamiltonian, with~$ω=1$, has eigenvalue~$½$ in 1D and~$1$ in 2D.
My choice gives double those.

\bye