\input respnotes
\input xpmath
\input unifonts \tenrm

\def\cite#1{{\tt #1}}

\title Dimensionless units

Start with some identities for Gaussians.  The normal distribution
with standard deviation~$σ(t)$ is
$$φ(x,t) = {e^{-x²/2σ²}\over √{2π}σ}.$$
This is normalised with~$∫φ=1$.  It satisfies
$$φ_t = \left({x²\over σ²}-1\right){φ\over σ}\dot σ
	\qquad{\rm and}\qquad
	φ_{xx} = \left({x²\over σ²}-1\right){φ\over σ²}.
$$
So the heat equation~$φ_t=φ_{xx}$ has solution~$σ(t)=√{2t}$, which
satisfies~$\dot σ={1\over σ}$.

The corresponding wave function is
$$ψ(x) = {e^{-x²/2σ²}\over √{π}σ},$$
with~$∫ψ²=1$.  This satisfies
$$ψ_x = -{x\over σ²}ψ
	\qquad{\rm and}\qquad
	ψ_{xx} = {1\over σ²}\left({x²\over σ²}-1\right)ψ.
$$

I have been using a 1D harmonic oscillator Hamiltonian of~$H=-D_{xx}+x²$.
The normal choice is~$H=½(-D_{xx}+x²)$, obtained by setting~$ℏ=m$
and~$ω=1$.  The eigenstate is clearly the same in both cases,
with~$σ=1$.  This gives~$ψ₀(x)=π^{-1\over 4}e^{-x²/2}$, and, in 2D,
$ψ₀(x,y)=ψ₀(x)·ψ₀(y)={1\over √π}e^{-(x²+y²)/2}$.  The normal
hamiltonian, with~$ω=1$, has eigenvalue~$½$ in 1D and~$1$ in 2D.
My choice gives double those.

2020-03-09 Vector normalisation

While quantum physics is set up in terms of the~$L²$ norm, numerical
algorithms and subroutines use the~$l²$ norm.  It is convenient to
use the physics conventions when writing a program that implements
SOR from scratch, but when the program uses standard eigenvector
and optimisation routines extensively, it is clearer to use the
numerical convention consistently than to keep converting back and
forth.  So how does Schrödinger's equation look with a~$l²$ normalised
vector of collocation values?

Let the continuous wave function be~$ψ$ with~$∫|ψ|²=1$, and the
discrete values be~$φ_j$ with~$∑_j|φ_j|²=1$.  Dimensional analysis
suggests the conversion is~$φ_j=√hψ(x_j)$, which follows formally
from~$∑_jh|φ_j|²=h∫|ψ|²≈∫|ψ|²$.  Schrödinger's equation is
$$\eqalign{i{∂²φ\over ∂t²}&=\left(-{∇²\over 2}+V+C|ψ|²\right)φ\cr
    &=\left(-{∇²\over 2}+V+{C\over h}|ψ|²\right)φ.
}$$

\bye