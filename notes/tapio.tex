\input respnotes
\input xpmath
\input unifonts \tenrm

\def\cite#1{{\tt #1}}

\title History and goals of work with Tapio

There are a couple of science goals to this project.  An initial
one is to find out how the mass of a vortex, in the classical
dynamics model, depends on the radius at which it is orbiting the
condensate.  There is a “buoyancy force” on the vortex, pushing it
in the direction of lower atomic density, towards the edge of the
condensate.  Its mass can be determined from the speed at which it
orbits.  This is what the ARC has been told we will do.

The eventual goal is to run these simulations with fermions, and
work with Chris and the two-dimensional Dyscope.

\beginsection{Warming up}

Tapio has spent the last decade calculating condensate sound wave
modes in three dimensions.  This is numerically intensive.  The
excited modes of a Bose-Einstein condensate are very sensitive to
the condensate order parameter, so it must be relaxed to the
equilibrium value wiht six-figure accuracy.  He often had simulations
running for months on end, and the results of one simulation would
be analysed in multiple papers.

My initial goal was to repeat some of these calculations in 2D, and
confirm that my Julia code is giving the known results.  I started
by defining types for fields, derivative operators and mappings
from frequencies to mode functions.  At some point I decided to
start again from scratch, and see how simple this code be if it
just used matrices and vectors.  It turned out that the whole thing
fits in one screen, and this version was much more efficient than
the abstracted one.  At some point I'll develop a nice wrapper for
this, but it is useful to have a straightforward version to benchmark
it against.

\beginsection{Novel work}

There are several directions to take this.

Something that I can start at once is vortex lattice melting.  When
the atomic density decreases, the quantum fluctuations should cause
the vortices to form a liquid instead of a fixed lattice.  One sign
of this would be that the thermal iterations do not converge to a
self-consistent state, but keep finding different liquid states.
The first requirement here is to get a converged solid lattice.

One is to branch into fermions.  Chris Vale is building the Dyscope,
and he could use simulations of what he might do with it.

The core goal is dynamics.  The first physics problem is the Berry
phase of a single vortex orbiting in a trap.

Now that I have a good 2D Bogoliubov code, I could go back to the
Vienna trap problem.  The problems with the Bogoliubov ground state
and the full potential still apply.

\bye