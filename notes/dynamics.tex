\input respnotes
\input xpmath
\input unifonts \tenrm

\def\cite#1{{\tt #1}}
\def\\{\hat}
\def\hone{-{∇²\over 2}+V}
\def\Hone{\left(\hone\right)}

\title Bogoliubov sound wave dynamics

In parallel with implementing Bogoliubov dynamics numerically, I'm
trying to understand what they mean.

The standard Bose gas hamiltonian is
$$\\H = ∫\\ψ†\Hone\\ψ + {g\over 2}\\ψ^{2\dagger}\\ψ²\,dx.$$
Throughout this file, I'll assume that~$μ\\N$ has been absorbed
into~$V$.  When I do Bogoliubov dynamics, I'm representing~$\\ψ$ in
the form
$$\\ψ(x,t)=φ(x,t)\\a(t)+∑_ju_j(t,x)\\b_j(t)-v*_j(x,t)\\b†_j(t).$$
I'm going to suppose that this operator is in the Heisenberg picture.
It has time as an argument, so it looks that way, and the whole
idea of a dynamical order parameter makes more sense in the Heisenberg
picture, where I can defer the question of whether~$\\a$ acts
on a number state or a coherent state.

I'm hoping to diagonalise the Hamiltonian.  That is to say, I want
a set of modes whose annihilation operators have
dynamics~$b_j(t)=b_je^{iω_jt}$.  Later on, I might interpret~$\\ψ(x,t)$
as an interaction picture operator, and think about the dynamics
of the mode occupations.  But that's getting ahead of myself.

When~$\\ψ(x,t)$ is substituted into the Hamiltonian, I get
$$\\H(x,t)=φ*\Hone φ\\a†\\a + {g\over2}|φ|⁴\\a^{2\dagger}\\a²
	+ δ\\ψ†\\a†\left(\hone+g|φ|²\\a†\\a\right)φ\\a + {\rm h.c.}
	+ δ\\ψ†\Hone δ\\ψ
	+ {g\over2}\left(δ\\ψ^{2\dagger}φ²\\a² + {\rm h.c.}\right)
	+ 2g|φ|²\\a†\\a δ\\ψ†δ\\ψ
	+ O(δ\\ψ³).
$$

The first step is to reproduce the GPE from the~$\\a†\\a$ part.

Peter Drummond has a trick to get rid of the quartic term.

2019-08-27 BdG hermitianity and mode orthogonality 

\begingroup
\def\L{{\cal L}}

The BdG equations have various forms, many of them confusing.  In Fetter's original \cite{aop-70-67}, it's tempting to assume~$\L$ and~$\L*$ are hermitian conjugates.  In fact, they are both hermitian operators.  I'll go with Equation~2 of \cite{prl-92-060407}.  Note that this was a classical sound wave treatment, and doesn't consider the normalisation of~$ψ₀$ with respect to the~$u$ and~$v$ modes.

Let's identify a BdG mode with a column vector function~$|uv〉=\pmatrix{u&v}^{\rm T}$.  Then the eigenproblem is~$L|uv〉=ω|uv〉$, where the operator
$$L=\pmatrix{\L&-gψ²\cr gψ^{2\ast}&-\L*}$$
is not a hermitian matrix.  Instead, it satisfies~$L†=gLg†$, where~$g$ is the unitary matrix
$$g=\pmatrix{1&0\cr0&-1}.$$
You get back where you started by conjugating the matrix, then flipping the sign of the second row and the second column.  (The first column would do just as well.)

If the inner product of BdG modes is defined with the Lorentz metric, 
$$〈uv|u'v'〉=∫\pmatrix{u*&v*}·g·\pmatrix{u'\cr v'}=∫u*u'-∫v*v',$$
the linear operator~$L$ satisfies
$$(L|uv〉)†|u'v'〉=\left(L\pmatrix{u\cr v}\right)†·g·\pmatrix{u'\cr v'}
	=\pmatrix{u*&v*}·L†g·\pmatrix{u'\cr v'}
	=\pmatrix{u*&v*}·gL·\pmatrix{u'\cr v'}=〈uv|(L|u'v'〉),
$$
and the operator~$L$ is hermitian even though the matrix is not.

If~$|uv〉$ and~$|u'v'〉$ are eigenvectors of~$L$, with eigenvalues~$ω$ and~$ω'$, then
$$ω*〈uv|u'v'〉=ω'〈uv|u'v'〉.$$
It follows that, {\it unless~$|uv〉$ is a zero mode,} the eigenvalue
is real.  Also, different modes must be orthogonal unless their
eigenvalues are conjugate.

Note that adding a chemical potential~$μ$ to~$H$ changes~$L$ to~$L+μg$.  Then
$$〈uv|L+μg|uv〉=ω+μ〈uv|uv〉₊,$$
where~$+$ denotes the Euclidean inner product.  This can set the
real part of~$ω$ to zero, but does not change the imaginary part.

It's tempting to use Wick rotation to make this hermitian in Euclidian
space.  However, Wick rotation only works for real vectors; it has
no effect on complex vectors with a conjugate-linear inner product.

\endgroup

2019-11-20

The orbiting vortex is not static, but there is some sense in which
the order parameter changes slowly.  The papers \cite{pra-68-033611}
and \cite{prl-87-230403} use a criterion~$∥{∂ψ\over ∂t}∥/∥ψ∥$ to
evaluate this.  For example, they can vary the radius of an offset
vortex, and see how this changes.  This value is very similar to
the~$〈μ〉$ that Tapio and I have talked about; what would be really
interesting is the residual $(∂\over ∂t}-〈μ〉)ψ$.  That's the
residual that SOR is trying to minimise.  It would be interesting
to see how well SOR minimises it, and if other methods do any better.

\bye