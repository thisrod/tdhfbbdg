\input respnotes
\input xpmath
\input unifonts \tenrm

\def\cite#1{{\tt #1}}

\title Berry phase

2019-10-21

A quantum system whose hamiltonian is independent of time has dynamics
$$|ψ(t)〉=∑_n |ψ_n〉e^{-iω_nt},$$
where~$|ψ_n〉$ is the eigenstate with energy~$ℏω_n$.  Things are
different when the hamiltonian changes with time.  However, there
are still eigenstates such that
$$\hat H(t)|ψ_n(t)〉=ℏω_n(t),$$
and in the adiabatic limit, I might guess that
$$|ψ(t)〉=∑_n |ψ_n〉e^{-iφ_n(t)},$$
where~$\dot φ_n=ω_n$.

In fact, for some “topological” systems, the phase picks up an extra term, 
$$φ_n = ∫ω_n + φ_c,$$
where, in the adiabatic limit, $φ_c$ only depends on the path drawn
out by the Hamiltonian and the topology of the parameter space.
The challenge with Andrew Groszek and Tapio is to define this
geometric phase when the time-dependent dynamics are caused by a
vortex moving in a trap.

2019-11-14

I'm reading Berry's paper, \cite{rsa-392-45}.  The key result is
that, where a hamiltonian~$H(R)$ depends on a set of parameters~$R$,
there is a “phase” two-form in parameter space, and the geometric
phase around a circuit is the integral of its exterior derivative
around the circuit/of it over any surface bounded by the circuit.

In the orbiting vortex case, the parameter is the order parameter~$ψ₀$
(or the vortex location, which determines the order parameter).
The hamiltonian~$H(ψ₀)$ is presumably the Bogoliubov approximate
hamiltonian.  We could create a quasiparticle in any mode, then see
how the phase of that excited state changes as the vortex does a
lap of the trap.

There's a wierdness here.  The Bogoliubov hamiltonian is supposed
to be a good approximation to the atomic field hamiltonian, which
doesn't change as the vortex orbits.  Maybe there is always some
ambiguity as to what is a classical parameter and what is a property
of the quantum state.  So how does Berry phase work when you're
changing the quantum state?

2019-12-05

The Berry phase of a vortex is a bit strange.  Let~$ψ$ and~$φ$ be the same order parameter.  Then the GPE can be written
$$iℏψ_t=(-½∇²+V+C|φ|²)ψ.$$
This has the form of a linear Schrödinger equation for the wave function~$ψ$, with~$φ$ as a parameter.

In the strict Berry formalism, we're thinking about the eigenstates of the operator~$H_φ=-½∇²+V+C|φ|²$, as~$φ$ varies adiabatically.  That is all well defined, and no doubt gives the same answer as Haldane.  But if we know how the dynamics works, we can adjust the parameter~$φ(t)$ so that the dynamical~$ψ(t)=φ(t)$ is a solution to the GPE.  This~$ψ(t)$ can't be an exact eigenstate of~$H_{φ(t)}$, because it evolves in time, but if it evolves slowly it will be close to a eigenstate.

A reasonable program would be:

\item{1.} Solve the GPE for~$φ(t)$.

\item{2.} Find the eigenstates of~$H_{φ(t)}$ as a function of time, and their geometric phases.

\item{3.} Expand~$φ(t₀)$ over the eigenstates, and rotate the phases according to Berry.

\item{3.} Expand~$φ(t₁)$ over the eigenstates, and compare.



\bye