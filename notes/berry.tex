\input respnotes
\input xpmath
\input unifonts \tenrm

\def\cite#1{{\tt #1}}
\catcode`∮=\active
\def∮{\oint}

\title Berry phase and Bose gas vortices

2019-10-21

A quantum system whose hamiltonian is independent of time has dynamics
$$|ψ(t)〉=∑_n |ψ_n〉e^{-iω_nt},$$
where~$|ψ_n〉$ is the eigenstate with energy~$ℏω_n$.  Things are
different when the hamiltonian changes with time.  However, there
are still eigenstates such that
$$\hat H(t)|ψ_n(t)〉=ℏω_n(t),$$
and in the adiabatic limit, I might guess that
$$|ψ(t)〉=∑_n |ψ_n〉e^{-iφ_n(t)},$$
where~$\dot φ_n=ω_n$.

In fact, for some “topological” systems, the phase picks up an extra term, 
$$φ_n = ∫ω_n + φ_c,$$
where, in the adiabatic limit, $φ_c$ only depends on the path drawn
out by the Hamiltonian and the topology of the parameter space.
The challenge with Andrew Groszek and Tapio is to define this
geometric phase when the time-dependent dynamics are caused by a
vortex moving in a trap.

2019-11-14

I'm reading Berry's paper, \cite{rsa-392-45}.  The key result is
that, where a hamiltonian~$H(R)$ depends on a set of parameters~$R$,
there is a “phase” two-form in parameter space, and the geometric
phase around a circuit is the integral of its exterior derivative
around the circuit/of it over any surface bounded by the circuit.

In the orbiting vortex case, the parameter is the order parameter~$ψ₀$
(or the vortex location, which determines the order parameter).
The hamiltonian~$H(ψ₀)$ is presumably the Bogoliubov approximate
hamiltonian.  We could create a quasiparticle in any mode, then see
how the phase of that excited state changes as the vortex does a
lap of the trap.

There's a wierdness here.  The Bogoliubov hamiltonian is supposed
to be a good approximation to the atomic field hamiltonian, which
doesn't change as the vortex orbits.  Maybe there is always some
ambiguity as to what is a classical parameter and what is a property
of the quantum state.  So how does Berry phase work when you're
changing the quantum state?

2019-12-05

The Berry phase of a vortex is a bit strange.  Let~$ψ$ and~$φ$ be the same order parameter.  Then the GPE can be written
$$iℏψ_t=(-½∇²+V+C|φ|²)ψ.$$
This has the form of a linear Schrödinger equation for the wave function~$ψ$, with~$φ$ as a parameter.

In the strict Berry formalism, we're thinking about the eigenstates of the operator~$H_φ=-½∇²+V+C|φ|²$, as~$φ$ varies adiabatically.  That is all well defined, and no doubt gives the same answer as Haldane.  But if we know how the dynamics works, we can adjust the parameter~$φ(t)$ so that the dynamical~$ψ(t)=φ(t)$ is a solution to the GPE.  This~$ψ(t)$ can't be an exact eigenstate of~$H_{φ(t)}$, because it evolves in time, but if it evolves slowly it will be close to a eigenstate.

A reasonable program would be:

\item{1.} Solve the GPE for~$φ(t)$.

\item{2.} Find the eigenstates of~$H_{φ(t)}$ as a function of time, and their geometric phases.

\item{3.} Expand~$φ(t₀)$ over the eigenstates, and rotate the phases according to Berry.

\item{3.} Expand~$φ(t₁)$ over the eigenstates, and compare.

2019-12-09

Ideas from Leggett's notes.

Berry's paper \cite{rsa-392-45} treats the case of Berry phase near
a degeneracy.  In the case that the ground state actually is
degenerate, the Berry phase is not just a complex number, but it
can become an element of a non-abelian group.  This is where braid
groups and topological quantum computing come from.

Topological things works better in 2D because you can say for sure
whether one point particle has gone around another.  This is another
way of saying that the homology is non-trivial.  In 3D, you can say
that a point has gone around a line, but how many laps it has done
around another point is ambiguous.

2019-12-10 Discretisation

Berry's formula 6 in \cite{rsa-392-45} is
$$γ(C)=i∫_C〈n|∇n〉·dR,$$
where~$C$ is a closed path in parameter space, and the gradient is
taken with respect to the parameter set~$R$.

Partition the path~$C$ as~$[R₀,R₁,…,R_m]$, and let~$|n_j〉≡|n(R_j)〉$.  Then, on~$[R_j,R_{j+1}]$, 
$$〈n|∇n〉·ΔR≈½(〈n_{j+1}|+〈n_j|)·(|n_{j+1}〉-|n_j〉)=i{\rm Im}〈n_j|n_{j+1}〉.
$$
The discrete
integral comes out to
$$γ(C)≈i∑_j〈n|∇n〉·ΔR≈-∑_j{\rm Im} 〈n_j|n_{j+1}〉
	=∑_j{\rm Im} 〈n_{j+1}|n_j〉.$$
On a closed path, with an analytic~$γ$, this type of approximation
is often spectrally convergent.

2019-12-18 Haldane's analysis

Haldane \cite{prl-55-2887} calculates the Berry phase for a vortex
in a BEC whose centre follows a path~$C$.  I have struggled to
reproduce this numerically, using the discretised Berry formula.
The crux of the calculation comes from \cite{prl-53-722}.  I intend
this to be a fully detailed derivation, with consistent notation
and conventions across the two papers.

Suppose that a bose gas has a single vortex core at~$R=(X,Y)$.  The
gas is wholly condensed, its state~$Ψ_R(x₁,y₁,…,x_N,y_N)=∏_jψ_R(x_j,y_j)$
being a product of single-boson wave functions, where the index~$j$
runs over the bosons.  The vortex traces a closed path~$C$, in the
course of which the wave function accumulates a Berry phase
$$\eqalign{γ&=i∮_C\left〈Ψ_R|∇_R|Ψ_R\right〉\,dR
	=i∮_C∑_j\left〈ψ_R|∇_R|ψ_R\right〉\,dR
	=Ni∮_C\left〈ψ_R|∇_R|ψ_R\right〉\,dR\cr
	&=Ni∮_CdR\,∫d²r\,ψ*_R(r)∇_Rψ_R(r).
}$$
To avoid the sums and products, the Berry phase can be
calculated for a single boson with wave function~$ψ_R$, then multiplied
by~$N$.  Alternatively, the wave function can be normalised with~$∫|ψ_R|²=N$
instead of~1, then substituted into the usual Berry formula.

Let the vortex-free wave function be~$φ(r)$.  In the Thomas-Fermi
regime, the wave function with a vortex centred at~$R$ has the form
$$ψ_R=A_Rf_Rφ.$$
Here~$A_Z(z)=z-Z$ is a phase singularity, and~$f_R(r)=g(|r-R|/ξ)$
is a radial density correction.  I will use the notation~$r=(x,y)$,
$z=x+iy$, $R=(X,Y)$ and~$Z=X+iY$, with the convention that $∇_R$~is
the gradient of a complex-valued function of two real variables,
while $d\over dz$ is a complex derivative.

To keep track of which quantities are being differentiated, and to
avoid writing out the components of vectors, let~$R=R(t)$.  Then
$${dψ_{R}\over dt}={d\over dt}A_{R(t)}f_{R(t)}φ
	=-\left({1\over z-Z}{dZ\over dt}+B(t)·{dR\over dt}\right)ψ_{R(t)}
$$
Where
$$B(t)={g'\over gξ}\left({∥r-R∥\over ξ}\right){r-R\over ∥r-R∥}.$$
Therefore
$$γ=-i∮_Cdt\,∫d²r\,\left({|ψ_R|²\over z-Z}{dZ\over dt}+|ψ_R|²B(t)\right).$$
The term in~$B(t)$ can be ignored, because~$B$
and~$γ$ are both real quantities.

Now split the boson density as~$|ψ_R|²=ρ₀+δρ_R$, where~$ρ₀$ is the
density of a vortex-free condensate, and~$δρ_R$ is the density
deficit due to the vortex core.  This splits~$γ$ into two corresponding
terms.  The first term involves~$ρ₀$, which is independent of~$t$
and can be factored out to make the line integral a contour integral
$$γ₀=i∫d²r\,ρ₀(r)∮_CdZ\,{1\over Z-z}.$$
In the case that the vortex makes an clockwise circuit, the contour
integral is~$-2πi$ for~$z$ inside the contour, and~$0$ otherwise.
(An anti-clockwise circuit has opposite sign.)  So in the clockwise
case,
$$γ₀=2π∫_Cd²r\,ρ₀(r).$$

For a given vortex position, the density deficit is dominated by a
symmetric term~$δρ_R(r)≈δρ₀(∥r-R∥)$.  But
$$∫d²r\,{δρ₀(∥r-R∥)\over z-Z}=0,$$
so the second term is the small quantity
$$γ₁=-i∮_CdZ\,∫d²r\,{δρ_R(r)-δρ₀(∥r-R∥)\over z-Z}.$$

Comments

\item{1.} The density deficit~$δρ_R$ is not quite circularly symmetric
under the ansatz~$ψ_R=A_rf_Rφ$, because~$φ$ might have a density
gradient.

\bye