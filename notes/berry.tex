\input respnotes
\input xpmath
\input unifonts \tenrm

\def\cite#1{{\tt #1}}
\catcode`∮=\active
\def∮{\oint}

\title Berry phase and Bose gas vortices

2019-10-21

A quantum system whose hamiltonian is independent of time has dynamics
$$|ψ(t)〉=∑_n |ψ_n〉e^{-iω_nt},$$
where~$|ψ_n〉$ is the eigenstate with energy~$ℏω_n$.  Things are
different when the hamiltonian changes with time.  However, there
are still eigenstates such that
$$\hat H(t)|ψ_n(t)〉=ℏω_n(t),$$
and in the adiabatic limit, I might guess that
$$|ψ(t)〉=∑_n |ψ_n〉e^{-iφ_n(t)},$$
where~$\dot φ_n=ω_n$.

In fact, for some “topological” systems, the phase picks up an extra term, 
$$φ_n = ∫ω_n + φ_c,$$
where, in the adiabatic limit, $φ_c$ only depends on the path drawn
out by the Hamiltonian and the topology of the parameter space.
The challenge with Andrew Groszek and Tapio is to define this
geometric phase when the time-dependent dynamics are caused by a
vortex moving in a trap.

2019-11-14

I'm reading Berry's paper, \cite{rsa-392-45}.  The key result is
that, where a hamiltonian~$H(R)$ depends on a set of parameters~$R$,
there is a “phase” two-form in parameter space, and the geometric
phase around a circuit is the integral of its exterior derivative
around the circuit/of it over any surface bounded by the circuit.

In the orbiting vortex case, the parameter is the order parameter~$ψ₀$
(or the vortex location, which determines the order parameter).
The hamiltonian~$H(ψ₀)$ is presumably the Bogoliubov approximate
hamiltonian.  We could create a quasiparticle in any mode, then see
how the phase of that excited state changes as the vortex does a
lap of the trap.

There's a wierdness here.  The Bogoliubov hamiltonian is supposed
to be a good approximation to the atomic field hamiltonian, which
doesn't change as the vortex orbits.  Maybe there is always some
ambiguity as to what is a classical parameter and what is a property
of the quantum state.  So how does Berry phase work when you're
changing the quantum state?

2019-12-05

The Berry phase of a vortex is a bit strange.  Let~$ψ$ and~$φ$ be the same order parameter.  Then the GPE can be written
$$iℏψ_t=(-½∇²+V+C|φ|²)ψ.$$
This has the form of a linear Schrödinger equation for the wave function~$ψ$, with~$φ$ as a parameter.

In the strict Berry formalism, we're thinking about the eigenstates of the operator~$H_φ=-½∇²+V+C|φ|²$, as~$φ$ varies adiabatically.  That is all well defined, and no doubt gives the same answer as Haldane.  But if we know how the dynamics works, we can adjust the parameter~$φ(t)$ so that the dynamical~$ψ(t)=φ(t)$ is a solution to the GPE.  This~$ψ(t)$ can't be an exact eigenstate of~$H_{φ(t)}$, because it evolves in time, but if it evolves slowly it will be close to a eigenstate.

A reasonable program would be:

\item{1.} Solve the GPE for~$φ(t)$.

\item{2.} Find the eigenstates of~$H_{φ(t)}$ as a function of time, and their geometric phases.

\item{3.} Expand~$φ(t₀)$ over the eigenstates, and rotate the phases according to Berry.

\item{3.} Expand~$φ(t₁)$ over the eigenstates, and compare.

2019-12-09

Ideas from Leggett's notes.

Berry's paper \cite{rsa-392-45} treats the case of Berry phase near
a degeneracy.  In the case that the ground state actually is
degenerate, the Berry phase is not just a complex number, but it
can become an element of a non-abelian group.  This is where braid
groups and topological quantum computing come from.

Topological things works better in 2D because you can say for sure
whether one point particle has gone around another.  This is another
way of saying that the homology is non-trivial.  In 3D, you can say
that a point has gone around a line, but how many laps it has done
around another point is ambiguous.

2019-12-10 Discretisation

Berry's formula 6 in \cite{rsa-392-45} is
$$γ(C)=i∫_C〈n|∇n〉·dR,$$
where~$C$ is a closed path in parameter space, and the gradient is
taken with respect to the parameter set~$R$.

Partition the path~$C$ as~$[R₀,R₁,…,R_m]$, and let~$|n_j〉≡|n(R_j)〉$.  Then, on~$[R_j,R_{j+1}]$, 
$$〈n|∇n〉·ΔR≈½(〈n_{j+1}|+〈n_j|)·(|n_{j+1}〉-|n_j〉)=i{\rm Im}〈n_j|n_{j+1}〉.
$$
The discrete
integral comes out to
$$γ(C)≈i∑_j〈n|∇n〉·ΔR≈-∑_j{\rm Im} 〈n_j|n_{j+1}〉
	=∑_j{\rm Im} 〈n_{j+1}|n_j〉.$$
On a closed path, with an analytic~$γ$, this type of approximation
is often spectrally convergent.

2019-12-18 Haldane's analysis

Haldane \cite{prl-55-2887} calculates the Berry phase for a vortex
in a BEC whose centre follows a path~$C$.  I have struggled to
reproduce this from first principles, using the discretised Berry
formula.  The crux of the calculation comes from \cite{prl-53-722},
whose notation and conventions differ from Haldane.  This will be
a detailed and consistent derivation.

Haldane treated a gas with multiple vortices, but my analysis is
restricted to one.

The entire gas is condensed, and so its state is a product of
single-boson states, $Ψ_R(x₁,y₁,…,x_N,y_N)=∏_jψ_R(x_j,y_j)$,
where the vortex core is located at~$R=(X,Y)$.  Sums and products
run over the $N$~bosons.  When~$R$ traces out a closed
path~$C$, the Berry phase is
$$\eqalign{γ&=i∮_C\left〈Ψ_R{∂\over ∂R}Ψ_R\right〉\,dR\cr
	&=i∮_C∑_j\left〈ψ_R{∂\over ∂R}ψ_R\right〉\,dR\cr
	&=Ni∮_C\left〈ψ_R{∂\over ∂R}ψ_R\right〉\,dR
}$$
To avoid all those sums and products, the Berry phase can be
calculated for a single boson with wave function~$ψ_R$, then multiplied
by~$N$.  Alternatively, the wave function can be normalised with~$∫|ψ_R|²=N$
instead of~1, then substituted into the usual Berry formula.

Let the vortex-free wave function be~$φ(x,y)$.
The wave function with a vortex centred at~$R$ is
$$ψ_R=A_Rf_Rφ,$$
where~$A_{(X,Y)}(x,y)=(x-X+i(y-Y))=z-Z$ is a phase singularity,
and~$f_R(r)=g(|r-R|/ξ)$ is a radial density, with~$g(r)→\hbox{const}/|r|$
as~$r→∞$.

To calcuate it, Haldane throws to \cite{prl-53-722}.  That paper
considers the case~$f(r)=1$, but perhaps~$f$ can be absorbed
into~$ψ⁰$, because~$f≈1$ at the vortex core, where the action is.
That paper claims that
$${∂\over ∂R}A(R)ψ₀=∑_i{∂\over ∂R'}\ln(z_i-R')A(R)ψ₀.$$
{\bf To do: verify that.}  The sum can be converted to an integral
by use of the boson density operator~$ρ(z) = 〈ψ|∑_iδ(z-z_i)|ψ〉$,
giving
$$〈ψ{∂\over ∂R}ψ〉=〈ψ{∂\over ∂R}∑_i\ln(z_i-R)ψ〉
	=∫\,d²zρ_R(z){∂\over ∂R}\ln(z-R),
$$
This is where Haldane resumes at Equation~10, with a switch of
exterior derivative from~$dz$ to~${\bf\hat z}×d{\bf R}$.  He
replaces~$\ln(z_i-R')$ with~$\ln(|z_i-R'|)$, presumably because~$γ$
is known to be real, so the integral of~$i{\rm Arg}(z_i-R')$ must be
zero.

The density
$ρ_R$ includes a vortex hole at~$R$, and the next step is to rewrite
it as~$ρ_R(r)=ρ⁰(r)+δρ(r-R)$, then split out a part~$δρ₀$ from~$δρ$
that has circular symmetry, and looks like a vortex hole in a uniform
gas.

Equivalent arguments are given in each paper, using Cauchy's theorem
in \cite{prl-53-722} and Green's theorem in \cite{prl-55-2887}.
Since~$ρ⁰$ is independent of~$R$, the line integral merely picks
out the interior of~$C$, and the area integral gives the number of
particles in a gas without the vortex hole.  {\bf How do you define
that, when the only order parameter you have includes a vortex?}
The correction term can be argued to be small, because it only
depends on the part~$δρ-δρ₀$ that lacks circular symmetry.

\bye