\input respnotes
\input xpmath
\input unifonts \tenrm

\def\cite#1{{\tt #1}}

\title Bogoliubov theory for fermions

In a system of fermions, there is no order parameter.  Therefore
the only density is the thermal density.  Instead of a Gross-Pitaveskii
equation, there is a Hartree equation, which looks very similar.
The density can be initialised by a Thomas-Fermi method.  The fermion
density is nonlinear in the chemical potential, but this is well
understood.

The BdG problem is very similar, except that some things change
sign.  In the BdG matrix, the off diagonal parts have a pair
potential~$Δ$ in place of the condensate wave function.  This is
an anomalous density.  Apparently, the Popov approximation doesn't
work for fermions, and I'll need to use Hartree-Fock-Bogoliubov.

The fermion modes have a gap.  There are quantised angular momenta
as with bosons, but a large density of states in each angular
momentum above the gap.  There is no Kelvin mode, so the convergence
is better: small changes in the Kelvin mode energy cause big changes
in the thermal density.

2019-10-22

The BdG equations for fermions have the form
$$\left(-{1\over 4E_f}∇²-E_f\right)u_j + Δv_j = ω_ju_j$$
and 
$$-\left(-{1\over 4E_f}∇²-E_f\right)v_j + Δ*u_j = ω_jv_j.$$

An obvious idea is to set~$Δ=0$, and iterate to self-consistency.
Unfortunately, the BdG problem has solutions~$Δ=0$, $u_j=φ_j$,
$v_j=0$, where~$φ_j$ is an eigenstate of the hamiltonian, and
similarly~$u_j=0$, $v_j=φ_j$, and this is a fixed point of~$Δ$.

Tapio suggests the alternative of setting~$Δ$ to the density of an
ideal fermi gas.  In my units, this is~$n={1\over 4π²}(E_f-V)^{3/2}$.
That seems to work.

\bye